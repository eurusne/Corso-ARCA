\begin{titlepage} 
\begin{center}
% Upper part of the page
\includegraphics[width=0.3\textwidth]{logoRed.eps}
\\
\vspace{2.0cm}
\textsc{\Large Università degli Studi di Padova}\\
\vspace{0.50cm}
\textsc{\large Dipartimento di Psicologia Generale}\\
\vspace{0.50cm} 
\textsc{\large Neuroscienze e Riabilitazione Neuropsicologica}\hspace{0.8cm}\\
  
% Title
\vspace{2.0cm}
\huge \doublespacing \bfseries \begin{spacing}{1}{Titolo della tesi}\end{spacing}
\vspace{2.0cm}
\vfill
\begin{minipage}[c]{0.48\linewidth}
    \begin{flushleft} \large
    \emph{Relatore:} \\
    Prof. Vicovaro \textsc{Michele}
\end{flushleft}
\end{minipage}%
\hfill%
\begin{minipage}[c]{0.48\linewidth}
    \begin{flushright} \large
    \emph{Laureanda:}\\
    Andrea \textsc{Nazzaro}

    \textsc{N. 2157172}
    \end{flushright}
    \vfill
\end{minipage}


% \hfill
\vfill
\vspace{2.00cm}
 
% Bottom of the page
{\small Anno accademico 2024/2025} 
\end{center}


\pagenumbering{gobble}


\newpage
\thispagestyle{empty}
%\null                                     % This command is weird. It's basically an invisible thing that gives sense to the commands. In this case
%\newpage                             % i wanted a white page after the title page, so i can have the index on a right page. Two \newpage 
                                           % commands wont do the job because they will be seen as just one from Latex. The \null command
                                           % emulates the presence of something to justify a new page.

\newpage
\null                         % This command is weird. It's basically an invisible thing that gives sense to the commands. In this case


\end{titlepage}
\pagestyle{plain}
\pagenumbering{roman}        % Roman page numbers (and reset to 1)
\tableofcontents                   % Make the Index based on the names of the sections/subsections/subsubsections

\newpage


\null
\newpage
\pagenumbering{arabic}        % Roman page numbers (and reset to 1)
\restoregeometry